\documentclass[sigconf]{acmart}

%% 刪除ACM Reference Format信息
\settopmatter{printacmref=false} % Removes citation information below abstract
\renewcommand\footnotetextcopyrightpermission[1]{} % removes footnote with conference information in first column
\pagestyle{plain} % removes running headers

\usepackage{xeCJK}
\usepackage{subfigure}
\usepackage{graphicx}
\usepackage{array}
\usepackage{enumitem}
\usepackage{multicol}
\usepackage{algorithm,algorithmic}
\usepackage{color}  
%\definecolor{shadecolor}{named}{Gray}  
\definecolor{shadecolor}{rgb}{0.92,0.92,0.92}  
\usepackage{framed}
\usepackage{ulem}
\usepackage{listings}
\usepackage{verbatim}

%% Font
\CJKfontspec{Noto Serif CJK TC} %思源宋體

%% ----------
\begin{document}

\title{離散數學HW03}

\author{班級:\underline{資工一}、學號:\underline{112590022}、姓名:\underline{陳駿逸}}
\orcid{}
\affiliation{%
  \institution{}
  \city{}
  \country{}
}

\maketitle



%% ----- 作業繳交說明 -----
\section{題目}
\subsection{作業繳交說明}
\begin{shaded}
\begin{itemize}
    \item[-] 從 overleaf 下載 hw03.zip
    \item[-] 從 overleaf 下載 hw03.pdf
    \item[-] 建立一個新檔案夾,命名如下
    \item[-] \color{red}\begin{verbatim}離散數學_111590XXX_姓名_HW03\end{verbatim}
    \item[-] \color{black}放入 hw03.zip
    \item[-] 放入 hw03.pdf
    \item[-] 將此檔案夾,壓縮成 zip 檔,檔名如下
    \item[-] \color{red}\begin{verbatim}離散數學_111590XXX_姓名_HW03.zip\end{verbatim}
    \item[-] \color{black}將此壓縮檔,上傳到北科 i 學園 PLUS
    \item[-] Do not share your homework with any living creatures.
    \item[-] 作業遲交以零分計。
\end{itemize}
\end{shaded}

%% ----- Question -----
\subsection{題號}
\begin{shaded}
\begin{itemize}
	%\item[-] page 214, chapter 3.1 Exercises 24
        \item[-] page 226, chapter 3.2 Example 9
	\item[-] page 228, chapter 3.2 Exercises 2
	\item[-] page 241, chapter 3.3 Exercises 2
\end{itemize}
\end{shaded}

%% ----- Problem -----
\section{作答}

% \subsection{page 214, chapter 3.1 Exercises 24}
% \begin{shaded}
%     Describe an algorithm that determines whether a function from a finite set to another finite set is one-to-one.
% \end{shaded}
% \begin{algorithm}[H]
%     \algsetup{linenosize=\tiny}
%     \scriptsize
%     \begin{algorithmic}[1]
%         \FOR{$i : = 1$ to $m$}
%         \STATE {$hit(b_i) := 0$}
%         \ENDFOR
%         \STATE {$one\_one := \mathbf{true}$}
%         \FOR{$j : = 1$ to $n$}
%         \IF{$hit(f(a_j)) = 0$}
%         \STATE {$hit( f(a_j) ) := $ \underline{~~請作答~~}}
%         \ELSE
%         \STATE{$one\_one := \underline{~~請作答~~}$}
%         \ENDIF
%         \ENDFOR
%         \RETURN \underline{~~請作答~~}
%     \end{algorithmic}
%     \caption{\footnotesize \newline procedure $one\_one$($f$ : function,$a_1,a_2,...,a_n,b_1,b_2,...,b_m$): integers)}
%     \label{alg:seq}
% \end{algorithm}

\subsection{page 228, chapter 3.2 Exercises 2}
\begin{shaded}
    Give a big-$O$ estimate for $f(n) = 3n \log(n!) + (n^2 + 3) \log n$, where n is a positive integer.
\end{shaded}
\uline{$n!=1\cdot2\cdot3\cdot...\cdot n < n\cdot n\cdot n\cdot...\cdot n=n^n$\newline}
\uline{$\log(n!) < \log(n^n)=n\log n$\newline}
\uline{$\because $by the theorem: $ 3n\log(n!)$ is $O(n^2\log n)$ and $(n^2 + 3) \log n$ is $O(n^2\log n)$}
\uline{$\therefore O(n^2\log n)$\newline}



\subsection{page 228, chapter 3.2 Exercises 2}
\begin{shaded}
    Determine whether each of these functions is $O(x^2)$.
    \begin{enumerate}[label=(\alph*)]
    	\item $f(x) = 17x + 11$
    	\item $f(x) = x^2 + 1000$
    	\item $f(x) = x log x$
    	\item $f(x) = x^4 / 2$
    	\item $f(x) = 2^x$
    	\item $f(x) = \lfloor x \rfloor \cdot \lceil x \rceil$
    \end{enumerate}
\end{shaded}
\begin{enumerate}[label=(\alph*)]
	\item Yes, $C = 18, k = 11$.
	\item \underline{Yes, C=2,k=$\sqrt {1000}$}
	\item \underline{Yes, C=1,k=1}
	\item \underline{No}
	\item \underline{No}
	\item \underline{Yes, C=2,k=1}
\end{enumerate}


\subsection{page 241, chapter 3.3 Exercises 2}
\begin{shaded}
    Give a big-O estimate for the number additions used in this segment of an algorithm.
    \begin{lstlisting}[language={python}]
        t := 0
        for i := 1 to n
            for j := 1 to n
                t := t + i + j
    \end{lstlisting}
\end{shaded}
The statement $t := t + i + j$ is executed \underline{$n^2$} times, so the number of operations is $O(\underline{n^2})$.


\clearpage

\vspace{20cm}

\end{document}
\endinput
%%
%% End of file `sample-sigconf.tex'.
