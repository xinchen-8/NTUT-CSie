\documentclass[sigconf]{acmart}

%% 刪除ACM Reference Format信息
\settopmatter{printacmref=false} % Removes citation information below abstract
\renewcommand\footnotetextcopyrightpermission[1]{} % removes footnote with conference information in first column
\pagestyle{plain} % removes running headers

\usepackage{xeCJK}
\usepackage{subfigure}
\usepackage{graphicx}
\usepackage{array}
\usepackage{enumitem}
\usepackage{multicol}
\usepackage{color}  
%\definecolor{shadecolor}{named}{Gray}  
\definecolor{shadecolor}{rgb}{0.92,0.92,0.92}  
\usepackage{framed}
\usepackage{ulem}

%% Font
\CJKfontspec{Noto Serif CJK TC} %思源宋體

%% ----------
\begin{document}

\title{離散數學HW02}

\author{班級:\underline{資工一}、學號:\underline{112590022}、姓名:\underline{陳駿逸}}
\orcid{}
\affiliation{%
  \institution{}
  \city{}
  \country{}
}

\maketitle

%% ----- 作業繳交說明 -----
\section{題目}
\subsection{作業繳交說明}
\begin{shaded}
\begin{itemize}
    \item[-] 從 overleaf 下載 hw02.zip
    \item[-] 從 overleaf 下載 hw02.pdf
    \item[-] 建立一個新檔案夾,命名如下
    \item[-] \color{red}\begin{verbatim}離散數學_111590XXX_姓名_HW02\end{verbatim}\color{black}
    \item[-] 放入 hw02.zip
    \item[-] 放入 hw02.pdf
    \item[-] 將此檔案夾,壓縮成 zip 檔,檔名如下
    \item[-] \color{red}\begin{verbatim}離散數學_111590XXX_姓名_HW02.zip\end{verbatim}\color{black}
    \item[-] 將此壓縮檔,上傳到北科 i 學園 PLUS
    \item[-] Do not share your homework with any living creatures.
    \item[-] 作業遲交以零分計。
\end{itemize}
\end{shaded}


%% ----- Question -----
\subsection{題號}
\begin{shaded}
\begin{itemize}
        \item[-] page 131, chapter 2.1 Exercises 2
	\item[-] page 132, chapter 2.1 Exercises 11
        \item[-] page 132, chapter 2.1 Exercises 36
	\item[-] page 144, chapter 2.2 Exercises 4
	\item[-] page 163, chapter 2.3 Exercises 38
	\item[-] page 179, chapter 2.4 Exercises 34
	\item[-] page 186, chapter 2.5 Exercises 2
 	\item[-] page 186, chapter 2.5 Exercises 6
  	\item[-] page 194, chapter 2.6 Exercises 20
\end{itemize}
\end{shaded}

%% ----- Problem -----
\section{作答}
\subsection{page 131, chapter 2.1 Exercises 2}
\begin{shaded}
    Use set builder notation to give a description of each of these sets.
    \begin{enumerate}[label=(\alph*)]
    	\item $\{0, 3, 6, 9, 12\}$
    	\item $\{-3,-2,-1, 0, 1, 2, 3\}$
            \item $\{m, n, o, p\}$
    \end{enumerate}
\end{shaded}  
\begin{enumerate}[label=(\alph*)]
	\item $\{3n\ |\ n = 0, 1, 2, 3, 4\}\ or\ \{x\ |\ x\ is\ a\ multiple\ of\ 3\wedge 0\le x\le 12\}$
	\item \underline{$\{x\in \mathbb{R}|-3\leq\ x \leq 3\}$}
    \item \underline{$\{$x is a letter between l and q$\}$}
\end{enumerate}

\subsection{page 132, chapter 2.1 Exercises 11}
\begin{shaded}
    Determine whether each of these statements is true or false.
    \begin{enumerate}[label=(\alph*)]
    	\item $0 \in \emptyset$
    	\item $\emptyset \in \{0\}$
            \item $\{0\} \subset \emptyset$
            \item $\emptyset \subset \{0\}$
            \item $\{0\} \in \{0\}$
            \item $\{0\} \subset \{0\}$
            \item $\{\emptyset\} \subseteq \{\emptyset\}$
    \end{enumerate}
\end{shaded}  
\begin{enumerate}[label=(\alph*)]
	\item False
	\item $\text{\uline{False}}$
 	\item $\text{\uline{False}}$
  	\item $\text{\uline{True}}$
   	\item $\text{\uline{False}}$
        \item $\text{\uline{False}}$
        \item $\text{\uline{True}}$
\end{enumerate}

\subsection{page 132, chapter 2.1 Exercises 36}
\begin{shaded}
    Find $A^3$ if
    \begin{enumerate}[label=(\alph*)]
    	\item $A = \{a\}$
    	\item $A = \{0, a\}$
    \end{enumerate}
\end{shaded}  
\begin{enumerate}[label=(\alph*)]
	\item $\{(a, a, a)\}$
	\item \underline{$\{(0,0,0),(0,0,a),(0,a,0),(0,a,a),(a,0,0),(a,0,a),(a,a,0),(a,a,a)\}$}
\end{enumerate}

\subsection{page 144, chapter 2.2 Exercises 4}
\begin{shaded}
    Let $A = \{a, b, c, d, e\}$ and $B = \{a, b, c, d, e, f, g, h\}$. Find
    \begin{enumerate}[label=(\alph*)]
    	\item $A \cup B$
    	\item $A \cap B$
    	\item $A - B$
    	\item $B - A$
    \end{enumerate}
\end{shaded}  
\begin{enumerate}[label=(\alph*)]
	\item $\{a, b, c, d, e, f, g, h\}$
	\item \underline{$\{a,b,c,d,e\}$}
	\item \underline{$\emptyset$}
	\item \underline{$\{f,g,h\}$}
\end{enumerate}

\subsection{page 163, chapter 2.3 Exercises 38}
\begin{shaded}
    Find $f \circ g$ and $g \circ f$ , where $f(x) = x^2 + 1$ and $g(x) = x + 2$, are functions from \textbf{R} to \textbf{R}.
\end{shaded}  

\begin{enumerate}[label=(\alph*)]
\item \underline{$f\circ g (x)=f(g(x))=f(x+2)=(x+2)^2+1$}
\item \underline{$g\circ f (x)=g(f(x))=g(x^2+1)=(x^2+1)+2$}
\end{enumerate}

\subsection{page 179, chapter 2.4 Exercises 34}
\begin{shaded}
    Compute each of these double sums.
    \begin{enumerate}[label=(\alph*)]
    	\item $\sum\limits_{i=1}^3 \sum\limits_{j=1}^2 (i-j)$
    	\item $\sum\limits_{i=0}^3 \sum\limits_{j=0}^2 (3i+2j)$
    	\item $\sum\limits_{i=1}^3 \sum\limits_{j=0}^2 j$
    	\item $\sum\limits_{i=1}^2 \sum\limits_{j=0}^3 (i^2j^3)$
    \end{enumerate}
\end{shaded}  
\begin{enumerate}[label=(\alph*)]
	\item $(1-1)+(1-2)+(2-1)+(2-2)+(3-1)+(3-2) = 3$
	\item \uline{$2+4+3+3+2+3+4+6+6+2+6+4+9+9+2+9+4= 78$}
	\item \uline{$0+1+2+0+1+2+0+1+2=9$}
	\item \uline{$1\cdot 0+1\cdot 1+1\cdot 8+1\cdot 27+4\cdot 0+4\cdot 1+4\cdot 8+4\cdot 27 =180 $}
\end{enumerate}

\subsection{page 186, chapter 2.5 Exercises 2}
\begin{shaded}
    Determine whether each of these sets is finite, countably infinite, or uncountable. For those that are countably infinite, exhibit a one-to-one correspondence between the set of positive integers and that set.
    \begin{enumerate}[label=(\alph*)]
    	\item the integers greater than 10
    	\item the odd negative integers
    	\item the integers with absolute value less than 1,000,000
    	\item the real numbers between 0 and 2
            \item the set $A \times Z^+$ where $A = \{2, 3\}$
            \item the integers that are multiples of 10
    \end{enumerate}
\end{shaded}  
\begin{enumerate}[label=(\alph*)]
        \item This set is countably infinite. The integers in the set are 11, 12, 13, 14, and so on. We can list these numbers in that order, thereby establishing the desired correspondence. In other words, the correspondence is given by $1\leftrightarrow11$, $2\leftrightarrow12$, $3\leftrightarrow13$, and so on; in general $n\leftrightarrow(n + 10)$.
	\item \uline{This set is countably infinite. $\{x\in\mathbb{Z^+}|-(2x-1)\}$}
	\item \uline{This set is finite. $\{999999,999998,...0...-999998,-999999\}$}
	\item \uline{This set is uncountable. The number after the decimal point can be infinite and it's can't be written by the sequence(can't written by a list of order).}
 	\item \uline{This set is countably infinite. $\{x\in \mathbb{Z^+}|\{2x,3x\}\}$}
  	\item \uline{This set is finite.$\{0,\pm10,\pm20,\pm30...\}$}
\end{enumerate}

\subsection{page 186, chapter 2.5 Exercises 6}
\begin{shaded}
    Suppose that Hilbert’s Grand Hotel is fully occupied, but the hotel closes all the even numbered rooms for maintenance. Show that all guests can remain in the hotel.
\end{shaded}  
\uline{Make a function from the set of positive integers to odd positive integers.}
\uline{$f(x)=2x-1$, put the currently from room(n) into room(2n-1), thus room 1 is empty, and the guest in room 2 move to room 3 and the guest in room 3 move to room 4 and so on.}

\subsection{page 194, chapter 2.6 Exercises 20}
\begin{shaded}
    Let\\
    \[A = \left[\begin{matrix}
        -1 & 2 \\
        1 & 3 \\
        \end{matrix}\right].\]
    \begin{enumerate}[label=(\alph*)]
    	\item Find $A^{-1}$. [Hint: Use Exercise 19.]
    	\item Find $A^3$.
    	\item Find $(A^{-1})^3$.
            \item Use your answers to (b) and (c) to show that $(A^{-1})^3$ is the inverse of $A^3$.
    \end{enumerate}
\end{shaded}  
\begin{enumerate}[label=(\alph*)]
        \item Using Exercise 19, noting that $ad-bc = -5$, we write down the inverse immediately:
        \(\left[\begin{matrix}
            -\frac{3}{5} & \frac{2}{5} \\
            \frac{1}{5} & \frac{1}{5} \\
            \end{matrix}\right].\)
	\item \underline{$\begin{bmatrix}1&18\\9&37\end{bmatrix}$}
	\item \underline{$\begin{bmatrix}-\frac{37}{125}&\frac{18}{125}\\\frac{9}{125}&-\frac{1}{125}\end{bmatrix}$}
	\item \uline{They have same answer of$(A^{-1})^3$ and $(A^3)^{-1}$. We can make the inverse of (b), and put three power of (a), the matrices are equal. }
\end{enumerate}

\end{document}
\endinput
%%
%% End of file `sample-sigconf.tex'.
